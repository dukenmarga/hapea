\documentclass{article}
\usepackage{amsmath}
\usepackage{tgschola}

\begin{document}
	\begin{flushleft}
		Left align
	\end{flushleft}
	
	\begin{flushright}
		Right align
	\end{flushright}

	inline equation $a=2b+x$

	\begin{displaymath}
		x=2b+4^2
	\end{displaymath}
	shortcut for display math $$b=2x+c$$
	\begin{displaymath}		
		\prod_{i=1}^{\infty} a_{i}
	\end{displaymath}
	Default style inline math: $\prod_{i=1}^{\infty} a_{i}$
	$$	\prod_{i=1}^{\infty} a_{i} $$
	Inline display but enforce style: ${\displaystyle \prod_{i=1}^{\infty} a_{i}}$
	
	Contrary to popular belief, Lorem Ipsum is not simply random text.
	It has roots in a piece of classical Latin literature from 45 BC,
	making it over 2000 years old. Richard McClintock, a Latin professor
	at Hampden-Sydney College in Virginia, looked up one of the more
	obscure Latin words, consectetur, from a Lorem Ipsum passage, and
	going through the cites of the word in classical literature,
	discovered the undoubtable source. Lorem Ipsum comes from sections
	1.10.32 and 1.10.33 of "de Finibus Bonorum et Malorum"
	(The Extremes of Good and Evil) by Cicero, written in 45 BC.
	This book is a treatise on the theory of ethics, very popular
	during the Renaissance. The first line of Lorem Ipsum,
	"Lorem ipsum dolor sit amet..", comes from a line in section 1.10.32.

	\begin{tabular}{|lc|r|}
		\hline
		Left & Center & Right \tabularnewline
		\hline
		\hline
		D & E & F \tabularnewline
		\hline
		D & E & F \tabularnewline
		D & E & F \tabularnewline
		\hline
		
	\end{tabular}
	
\end{document}